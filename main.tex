\documentclass{beamer}
\usetheme{CambridgeUS}

\title{Assignment 5: Papoullis Text Book}
\author{Cherukupalli Sai Malini Mouktika}
\date{\today}
\logo{\large \LaTeX{}}

\usepackage{graphicx}
\usepackage{amsmath}
\usepackage{amssymb}
 \usepackage{listings}
    \usepackage{color}                                            
    \usepackage{array}                                            
    \usepackage{longtable}                                        
    \usepackage{calc}                                            
    \usepackage{multirow}                                         
    \usepackage{hhline}                                          
    \usepackage{ifthen}   

\usepackage{multicolrule}


   
%table commands   
\def\inputGnumericTable{}

\usepackage[latin1]{inputenc}                                 
\usepackage{caption} 
\captionsetup[table]{skip=3pt}  

\renewcommand{\thefigure}{\arabic{table}}
\renewcommand{\thetable}{\arabic{table}}       


\providecommand{\brak}[1]{\ensuremath{\left(#1\right)}}
\renewcommand\thesection{\arabic{section}}
\renewcommand\thesubsection{\thesection.\arabic{subsection}}
\renewcommand\thesubsubsection{\thesubsection.\arabic{subsubsection}}

\newcommand{\graph}{\noindent \textbf{Graph: }}
\newcommand{\calc}{\noindent \textbf{Calculations: }}
\numberwithin{equation}{subsection}

\renewcommand{\thetable}{\theenumi}
\usepackage{amsmath}
\setbeamertemplate{caption}[numbered]{}
\providecommand{\pr}[1]{\ensuremath{\Pr\left(#1\right)}}
\providecommand{\cbrak}[1]{\ensuremath{\left\{#1\right\}}}

\begin{document}

\begin{frame}
    \titlepage 
\end{frame}

\logo{}

\begin{frame}{Outline}
    \tableofcontents
\end{frame}
\section{Question}
\begin{frame}{Question}
    \begin{block}{Example 2.7}
    A telephone call occurs at random in the interval (0, T). This means that the probability that it will occur in the interval 0$\leq$t$\leq$to equals $\frac{to}{T}$. Thus the outcomes of this 
experiment are all points in the interval (0, T) and the probability of the event \{the call will occur in the interval (t1,t2)\} equals 


    \end{block}
\end{frame}
\section{Solution}
	\begin{frame}{Solution}
Given,

probability that it will occur in the interval 0$\leq$t$\leq$to = $\frac{to}{T}$. we know that 
\begin{align}
\int_{0}^{to} \alpha(t) \, dt = \frac{to}{T}
\end{align}
as we know that it is linear random variable $\alpha(t)$ is constant
by taking $\alpha(t)$ as $\alpha$ 
\begin{align}
\alpha\times(to-0)= \frac{to}{T}\\
\implies
\alpha = \frac{1}{T}
\end{align}
\end{frame}
\begin{frame}{}
 So,
\begin{align}
P\{t1 \leq t\leq t2\} &=  \int_{t1}^{t2} \alpha(t) \, dt\\
 &= \alpha\times(t2-t1)\\
 &= \frac{1}{T}\times(t2-t1)\\
 &=\frac{t2-t1}{T}
\end{align}
\end{frame}
\begin{frame}
  Thus the outcomes of this 
experiment are all points in the interval (0, T) and the probability of the event \{the call will occur in the interval (t1,t2)\} equals $\frac{t2-t1}{T}$  
\end{frame}
\end{document}